\pagestyle{headings}
\pagenumbering{arabic} 
\setcounter{page}{1}

\section{Introduction}

Microtubules (MT) mediate various cellular functions such as structural support, chromosome segregation, and intracellular transport. With this they are essential cytoskeletal elements. \\
They consist mainly out of $\alpha$- and $\beta$-tubulin. 
$\alpha$- and $\beta$-tubulin build the MT as a hollow tube while forming a $\alpha$- and $\beta$-tubulin dimer. This tube has a diameter of 25nm and consists out of a parallel arrangement of 13 $\alpha$, $\beta$-dimers single strands. The end of the MT with $\alpha$-tubulin is called the minus end and the other end of the MT with the $\beta$-tubulin gets called plus end. The minus end is less dynamic than the plus end. The plus end functions as the growth area of the MT by adding GTP-bounded tubulin dimers to it. The GTP gets hydrolysed to GDP after that. In the presence of GDP-bounded tubulin at the still growing plus end the MT collaps. This shrinkage is called 'catastrophe'. \\\\
The properties of the MT are based on the (1.) isoforms and (2.) the covalent posttranslational modifications (PTM) of the different tubulins, which is called the 'tubulin-code'.\\
Most PTMs occur at the C-terminal end of a tubulin amino-acid side chain. The PTMs can produce alterations in cellular function and its phenotype.\\
Acetylation of tubulins is one of the PTMs of the MT. Acetylation of K40 luminal $\alpha $-tubulin decreases the stiffness of MT and is correlated with a long MT lifetime because the reduced stiffness makes it more resilient against a mechanical stress force. \\
K40 acetylation is the only PTM within the lumen of MTs. 




\subsection{Aim}
In the first experiment, we transfected COS-7 with EB3-GFP DNA for live-imaging of the comet-like growth dynamics of the MT with a TIRFM.\\
In the second experiment, fixed microtubules were imaged for analysing the ratio of the acetylated MT. For this the cell were treated with the MT stabilising molecule Taxol and with DMSO as a control.