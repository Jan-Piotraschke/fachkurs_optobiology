\pagestyle{headings}
\pagenumbering{arabic} 
\setcounter{page}{1}

\section{Einleitung}

Microtubules (MT) mediate various cellular functions such as structural support, chromosome segregation, and intracellular transport. \cite{bar2022direct} They are essential cytoskeletal elements. \\
MTs consist mainly out of $\alpha$- and $\beta$-tubulin. 
% For the sake of completeness it should be mentioned that next to it other tubulin proteins like $\gamma$-, $\delta$- and $\epsilon$-tubulins also exist. \\
$\alpha$- and $\beta$-tubulin form the MT as a hollow tube while forming  $\alpha$- and $\beta$-tubulin dimers. The end of the MT with $\alpha$-tubulin is called the minus end and the other end of the MT with the $\beta$-tubulin get's called plus end. The minus end is less dynamic than the plus end because the plus end functions as the growth area of the MT by adding tubulin dimers to it. \\\\
The properties of the MT are based on the (1.) isoforms and (2.) the covalent posttranslational modifications (PTM) of the different tubulins, which is called the 'tubulin-code'.\\
The existence of 9 $\alpha$- and 10 $\beta$-tubulins isoforms in humans was verified.\\
Acetylation and detyrosination of tubulins are two of the PTMs of the MT. They are a reversible process and have a cross-talk between each other.\\
Acetylation of K40 luminal $\alpha $-tubulin decreases the stiffness of MT and is correlated with a long MT lifetime, because the reduced stiffness makes it more resilient against a mechanical stress force. Detyrosination of MT increases the stability of MT by decreasing the interaction of the MT with depolymerizing proteins. \\\\
K40 acetylation is the only PTM within the lumen of MTs. \\
Most PTMs occur at the C-terminal end of a tubulin amino-acid side chain. The PTMs  can produce alterations in cellular function and it's phenotype. \cite{seet2006reading} Detyrosination removes the C-terminal tyrosine of most $\alpha$-tubulin isoforms. \\
It has been shown that the VASH1/2:SVBP complex is enzymatic activate during the MT detyrosination.



