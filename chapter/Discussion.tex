\newpage
\section{Discussion}
The protein EB3-GFP is a well-established marker of growing MTs. EB family proteins are plus-end-interacting proteins. They autonomously recognize the structure of the MT growing plus ends. \\
If this EB3 comet is vanishing, a catastrophe happened for the MT.
The images of the EB3 comets were sampled with TIRFM. TIRFM stands for 'Total Internal Reflection Fluorescence Microscopy'.
The effect of total internal reflection is used to study the near-surface process. The evanescent wave has a penetration depth of approximately 100-200nm and only excites the fluorophores in this range. The method was ideal for our experiment because we used adherent COS-7 cells. \\ 
Unfortunately, the wrong DNA was used for the plasmid transfection and we couldn't visualize the EB3 comets in our experiment. With that, we couldn't calculate any statistics about it because we only had the given prime example of the practical course. 
\\\\
Immunocytochemistry (ICC) is a method that is used to visualize the localization of a specific protein or antigen in cells. It uses a specific primary antibody that binds to the protein or antigen. In our experiment we tagged the acetylated tubulin with a primary antibody and visualized it with the help of the fluorescent secondary antibody Alexa Fluor 568.\\
We combined this approach with the treatment of Taxol.
Taxol stabilizes the MTs. It inhibits the depolymerization of the MT inside the cells by binding to the $\beta$-tubulin. \cite{sackmann2010lehrbuch} The treatment time of 2h (Fig. \ref{acetylated_tubulin}) exemplifies this effect. With Taxol-treated cells, the ratio of the acetylated tubulin to total tubulin is double as much as compared to the control group of the DMSO-treated cells. Therefore, stabilized MT corresponds to a relative higher ratio of the acetylation PTM.
\\\\
It seems like there is a linear threshold for the occurrence of the catastrophe of MT (Fig. \ref{mt_live}). It would be interesting to know whether a comet has a minimal lifetime or the visualized threshold is the result of the sampling rate of the images, which can cut off some values. A higher sampling rate could answer this question.\\
Furthermore, acetylated MT should be more resilient to mechanical stresses. A experiment could get conducted where cells get exposed to this stress and the corresponding acetylated MT ratio could get analysed. 


\newpage